\setcounter{page}{1}
\section{Company}

\subsection{TVS Motors}

\textbf{Company Name}: TVS Motors\\
TVS stands for Thirukkurungudi Vengaram Sundram, which is the name of the founder of the group.

\textbf{Type of Company}: Large scale, 2,3 wheeler manufacturing\\
TVS Motor Company Limited (TVSM) \cite{TVS_ANNUAL_REPORT} is a globally recognized large-scale manufacturer of two and three-wheelers, operating in over 80 countries. The company ranks as the fourth-largest two-wheeler manufacturer in the world, with a diverse product range that includes scooters, motorcycles, mopeds, and electric vehicles. TVS is considered a large-scale company for several reasons:
\begin{enumerate}
	\item \textbf{Global Reach}: With a presence in more than 80 countries and state-of-the-art manufacturing facilities in India and Indonesia, TVSM serves a large international customer base.

	\item \textbf{High Production and Sales}: In FY 2023-24, TVS Motor achieved sales of over 4 million two-wheelers, which marks a significant scale of production.

	\item \textbf{Revenue}: The company reported a revenue of Rs. 31,925 crore in FY 2023-24, reflecting its large market presence and financial strength.
	
	\item \textbf{R \& D and Innovation}: TVS invests heavily in research and development, with a focus on future mobility solutions such as electric vehicles, evidenced by its innovative product lineup like the TVS iQube and the recently launched TVS X.

	\item \textbf{Sustainability Initiatives}: TVSM is also committed to sustainability, with 93\% of its energy requirements met through renewable sources.
\end{enumerate}

\textbf{Exports}:\\
 TVS Motor Company \cite{TVS_ANNUAL_REPORT} has established a robust global presence, exporting its two and three-wheelers to over 80 countries across Asia, Africa, Latin America, and Europe. The company has strategically expanded its international footprint, offering a range of products, including premium motorcycles and electric vehicles, such as the TVS Apache and TVS iQube. In FY 2023-24, TVS exported close to 0.89 million two-wheelers, demonstrating its resilience in global markets despite challenges like economic slowdowns in specific regions. With its entry into European markets through partnerships in France and Italy, TVS continues to strengthen its international distribution network and plans further expansion into select EU markets. Its growing focus on electric vehicles is set to drive future export growth, aligning with global trends towards sustainable mobility.

\textbf{Difference in Exports}:\\
 There are slight differences in exports based on the surrounding conditions of the country and based on the Government intervention of that country \cite{tvs_international}. There are some products which are extensively sold in some countries but barely sold in others based on the usage of that vehicle.

Products sold in Europe and in India are slightly different considering that Europe has a freezing cold period, hence the coolant used in the radiators of the vehicles are mixed with an antifreeze solution which is not done in India. Similarly, the famous TVS Auto (three wheeler) is extensively sold in India but barely sold in Europe.

\subsection{Hero MotoCorp}

\textbf{Company Name}: Hero MotoCorp

\textbf{Type of Company}: Large scale, 2 wheeler manufacturing

Hero MotoCorp is a large-scale company, as indicated by several factors across its reports:

\begin{enumerate}

	\item \textbf{World's Largest Two-Wheeler Manufacturer}: Hero MotoCorp has held the title of the world's largest two-wheeler manufacturer for 23 consecutive years, producing motorcycles and scooters. It has sold over 116 million units globally \cite{hero_sreport} \cite{hero-rep}.

	\item \textbf{Global Presence}: The company has a widespread presence, with operations in 47 international markets and manufacturing plants across eight locations in India and other countries such as Bangladesh and Colombia \cite{hero-rep}.

	\item \textbf{Scale of Operations}: Hero MotoCorp's national operations span 36 states, and it serves a significant number of customers across the globe. Its vast scale is reflected in both the volume of vehicles produced (over 100 million) and the number of employees and workers (around 30,000) \cite{hero-rep}

	\item \textbf{Financial Performance}: With revenue crossing Rs. 33,805.65 crore and net worth reaching Rs. 16,705.09 crore, Hero MotoCorp's financial scale underlines its large corporate status \cite{hero-rep}.

\end{enumerate}

These aspects demonstrate that Hero MotoCorp is a large-scale company, due to its global market leadership, operational scale, financial strength, and widespread geographical reach.

\textbf{Exports}: \\
Hero MotoCorp's export operations are an essential part of its global presence, contributing to 3.4\% of its total turnover \cite{hero_sreport}. The company serves 47 international markets across regions including Asia, Central and Latin America, Africa, and the Middle East\cite{hero_sreport}. With manufacturing units outside India, such as in Bangladesh and Colombia, Hero MotoCorp continues to expand its international footprint. Its strategy for growth includes the export of both motorcycles and scooters, catering to diverse customer needs worldwide. Despite a relatively smaller contribution to its total revenue from exports, Hero MotoCorp is actively increasing its presence in international markets, with significant expansion plans in electric vehicle segments\cite{hero_sreport}\cite{hero-rep}.

\textbf{Difference in Exports}:\\
This is similar to TVS Motors, the changes only occur due circumstances like the surrounding environment, Government intervention, commercial profits based on product, raw material availability.