\section{Performance evaluation}

\subsection{At TVS}

At TVS Motor Company, employee performance is measured and evaluated through a structured and comprehensive process that aligns individual goals with the company’s strategic objectives. The performance evaluation system takes into account multiple criteria, including achievement of business plans, contribution to the development of business strategy, and performance in relation to key responsibilities.

For senior managerial personnel, performance is primarily measured against the achievement of business plans approved by the Board during and after the financial year. This includes evaluating key deliverables like revenue, profitability, innovation, and operational excellence. Their annual performance incentives are directly linked to these performance metrics.

The Nomination and Remuneration Committee is responsible for overseeing the evaluation process and has established a peer evaluation methodology for assessing individual Directors and senior managers. This includes criteria such as strategic thinking, leadership, active participation in company initiatives, and contribution to Board cohesion. For employees at all levels, the focus is on how well they meet their role-specific objectives, adhere to company values, and contribute to team success. This robust evaluation process ensures transparency and accountability across the organization, fostering continuous improvement and alignment with corporate goals\cite{TVS_ANNUAL_REPORT}.

\begin{flushright}
	P.T.O
\end{flushright}

\newpage

\subsection{At Hero}

General corporate practices in performance management at large organizations like Hero MotoCorp\cite{hero-employee} typically involve structured processes such as:

\begin{enumerate}

	\item \textbf{Performance Appraisal}: Employees' performance is regularly assessed through formal performance appraisal systems, which evaluate their achievements against set goals. These appraisals often involve feedback from managers and peers.

	\item \textbf{Key Performance Indicators (KPIs)}: Employees are evaluated based on specific, measurable KPIs related to their roles. This ensures that their contributions align with the company's objectives.

	\item \textbf{360-Degree Feedback}: Many companies also use 360-degree feedback systems where input is collected from an employee's subordinates, colleagues, and supervisors to provide a holistic evaluation. But the Hero MotoCorp does not follow this method of 360 degree feedback.

	\item \textbf{Regular Reviews}: Performance is typically reviewed on a quarterly or annual basis, allowing for adjustments in targets and continuous improvement.
\end{enumerate}