\section{Milestones and Achievements}

\subsection{Achievements by TVS}

Some of the achievements of the TVS Motors is listed below in chronological order from 1980 to 2016. \cite{tvsmotorMotorCompanyAcheivements}  

\begin{itemize}
	\item 1980: Great milestone in Indian automobile history. Country's first 2 seater 50cc moped TVS 50 launched.
	\item 1984: First mover. TVS becomes the first Indian Company to introduce 100cc Indo-Japanese motorcycles.
	\item 1994:	Pioneer of mobility for women. Launched India's first indigenous Scooterette (sub-100 cc variomatic scooter), TVS Scooty.
	\item 1996 - 1997: Bringing in green technology before it became a norm. Introduced India's first catalytic converter enabled motorcycle, the 110cc Shogun. Great ride. Greater speed. Launched India's first 5-speed motorcycle, the Shaolin.
	\item 2000 - 2001: Hiking speed limits. Launched TVS Fiero, india's first 150 cc, 4 stroke motorcycle. Indigenous technology. Launched TVS Victor, 4-stroke 110 cc motorcycle India's first fully indigenous designed and manufactured motorcycle.
	\item 2002:	TVS becomes world's first two wheeler company to win world's most prestigious recognition in Total Quality Management- the Deming Award 2002. TVS wins Technology Award from Ministry of Science, Government of India for successful commercialization of indigenous technology.
	\item 2004:	Setting benchmarks in mileage. Launched TVS Centra, a world-class 4-stroke 100cc motorcycle with the revolutionary VT-i Engine for best-in-class mileage.
	All terrain performance. Launched TVS Star, a 100 cc motorcycle ideal for rough terrain.
	TVS wins TPM Excellence Award from Japan Institute of Plant Maintenance (JIPM).
	TVS wins Outstanding Design Excellence Award for TVS Scooty Pep.
	\item 2005 - 2006: Spreading its roots. TVS launches its Indonesian plant.
	Making a style statement. Launched TVS Apache, which set the youth's imagination on fire. Apache went on to win 6 prestigious awards
	\item 2007:
	Never before in automobile history. TVS Motor Company rolls out seven new products
	Yet another first. TVS launches its Himachal Pradesh Plant at Nalagarh.
	\item 2008:
	Apache Refresh with rear disc brakes launched in Dec-2008
	TVS Motor Company bags two coveted IT awards SAP ACE 2008 and 2008 Symantec South Asia Visionary Award
	Scooty Pep + launched with balancing wheels in Aug-2008
	Scooty Wimbeldon collection launched in Jun-2008
	Apache RTR Fi launched in Jun-2008
	TVS Motor Company launches the revolutionary 125cc FLAME in Mar-2008
	TVS makes its foray into the three-wheeler market with TVS KING in Mar-2008
	\item 2009 - 2010:
	TVS unveils the Apache RTR 180 ABS
	Jive : The Auto - Clutch bike launched
	Wego : First scooter with Body Balance Technology
	\item 2012 - 2013:
	TVS is India's most trusted brand in the 2 wheeler category.
	Source:Economic Times Most Trusted Brands Survey 2012
	TVS Motor Company launched Its premium executive 125cc motorcycle, TVS Phoenix
	TVS Jupiter launched on Sep 2013
	
	\item 2014 - 2015:
	TVS Motor Company launched a bike in the popular commuter segment, all the StaR city+
	TVS launches its all new premium commuter scooter, the TVS Wego 110cc.
	TVS launches its award winning scooter, the TVS Scooty Zest 110 with best in class features.
	To celebrate the 1st year anniversary of the Jupiter, TVS launches the special edition of the Jupiter with a new colour and additional features, to be sold in limited numbers.
	Launched All New Phoenix 125
	J.D Power Asia Pacific India Automotive Awards
	The most Appealing Executive Scooter - TVS Wego
	The most Appealing Premium Motorcycle - TVS Apache
	The most Appealing Economy Motorcycle - TVS Sport
	
	\item 2016:
	After decades of conquering the track, 2016 saw the launch of the new TVS Apache RTR 200 4V. The most powerful Apache yet.
\end{itemize}	

\subsection{Achievements by Hero MotoCorp}

Some of the achievements of the TVS Motors is listed below in chronological order from 1984 to 2022. \cite{herogroupHeroGroupAcheivements}

\begin{itemize}
	\item 19th January, 1984:
	Commencement of operations as a joint venture between Hero Cycles of India and Honda of Japan.
	
	\item 1994:
	The company plans to expand the Dharuhera plant to 240,000 units per annum and establish a new plant at Gurgaon Industrial Estate with a capacity of 150,000 units per annum.
	
	\item 1997:
	Hero Honda Motors Ltd (HHML) inaugurated a modern motorcycle plant in Gurgaon for Honda Super Cub production, marking a new era in India's motorcycle industry. Hero Motors, part of the Rs. 1,600 crore Hero group, also established a plant in Brazil for manufacturing Hero Winner scooters.
	
	\item 1998:
	Hero Honda explores scooters with Honda post Honda's exit from Kinetic Honda venture. In a five-year agreement, Kinetic Honda Motor retains technical support and market access after Honda's stake sale to Kinetic Engineering. The company introduces CBZ(ee), a groundbreaking bike with Transient Power Fuel Control (TPEC) system.
	
	\item 1999:
	Hero Honda and 20th Century Finance signed a MoU for Hero Honda motorcycle financing. The joint venture between Honda Motor’s and the Hero group aims to increase its market share to 38.6\%. Hero Motors forms a joint venture with Briggs Stratton for four-stroke engine development. Honda Motor Japan re-enters Indian scooter market and explores three-wheelers in collaboration with Hero Honda Motors.
	
	\item 2001:
	Hero Honda becomes the largest two-wheeler manufacturing company globally.
	
	\item 2002:
	The company is in the BSE Sensex top 30
	
	\item 2003:
	Hero Honda and SBI Cards launch a co-branded credit card. Collaborates with SISI to train diploma-holding unemployed youth. Implements a dealer credit system to cut receivables, saving Rs 100 crore.
	
	\item 2007:
	Yutaka Kudo appointed as Director of Hero Honda Motors from April 1, 2007. Hero Honda remains World No. 1 for the 6th year
	
	\item 2008:
	Inauguration of the Hero Honda Haridwar Plant.
	
	\item 2010:
	Board decides to terminate joint venture with Honda, allowing Hero Group to buy out Honda's 26\% stake.
	
	\item July 2011:
	Hero Honda Motors Limited changes its name to Hero MotoCorp Limited. Hero Honda rebrands as Hero MotoCorp, marking a new licensing agreement with Honda. The company receives an updated Certificate of Incorporation as Hero MotoCorp Limited.
	
	\item 2012:
	Hero MotoCorp forms key alliances: Erik Buell Racing for high-end bike technology, a European design and tech partner, and HDFC Bank for 6.99\% financing. Hero Investment Pvt. Ltd. merges with Hero MotoCorp Ltd.
	
	\item 2013:
	Munjal Family (founder) holds around 40\% equity shares, individual shareholders hold approx. 7.44\%, and foreign institutional investors hold approx. 30\%.
	Hero MotoCorp starts constructing a new plant, Global Parts Centre, and an innovative Research \& Design center. Achieving technological milestones, the company records over 1.2 million retail sales in Oct-Nov and introduces a unique 5-year warranty on all two-wheelers.
	
	\item 2014:
	Joint venture with Bangladesh's Nitol-Niloy Group leads to the establishment of HMCL Niloy Bangladesh Limited.
	
	\item 2014- 2015:
	Equity investments include acquiring a stake in Erik Buell Racing and investing in Ather Energy.
	
	\item 2015:
	HMCL Americas INC settles to acquire EBR's Consulting Business, while Hero MotoCorp launches operations in its first plant outside India in Villa Rica, Colombia.
	
	\item 2018:
	Hero MotoCorp begins constructing a new plant in Andhra Pradesh.
	
	\item 2021:
	Hero MotoCorp partners with Gilera Motors Argentina and also reached a production milestone of 100 million units.
	
	\item 2022:
	Hero MotoCorp expands in El Salvador, collaborates with BPCL for EV charging infrastructure, and partners with Nagpur Police on International Women's Day.
\end{itemize}